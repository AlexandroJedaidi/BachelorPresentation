\section{Numerischer Lösungsansatz}

\begin{frame}{Numerische Verfahren}
    \begin{itemize}
        \item<1-> Gesucht ist ein Verfahren zur Approximation einer Lösung des Anfangswertproblems
        \eqref{awp-erste-ordnung}
        \item<2-> Eine Möglichkeit sind hierbei die \textbf{Einschrittverfahren} und \textbf{Mehrschrittverfahren}.
    \end{itemize}
\end{frame}

\subsection{Einschrittverfahren}

\begin{frame}{Einschrittverfahren}
    \begin{itemize}
        \item<1-> Ein explizites \textbf{Einschrittverfahren} für das Anfangswertproblem \eqref{awp-erste-ordnung} auf
        einer Diskretisierung $D_{\text{h}}=\{t_i=t_0+ih, i=0,\dots,N\}$ des Zeitintervalls $D=[t_0,t_0+a]$ und der
        Schrittweite $h=\frac{a}{N}$ hat die Form
        \begin{align}
            \label{einschrittverfahren}
            u_0 &= x_0 \nonumber \\
            u_{i+1} &= u_i + h \Phi(t_i,u_i, h, f), \quad i=0,\dots,N.
        \end{align}
        \item<2-> Mit $K\subseteq \mathbb{R}^n$ heißt
        $\Phi: D_{\text{h}} \times \mathbb{R}^n \times \mathbb{R} \times
        C(D \times K, \mathbb{R}^n) \rightarrow \mathbb{R}^n$ \textbf{Inkrementfunktion}
        \item Die Inkrementfunktion $\Phi$ von impliziten Einschrittverfahren hängt zusätzlich von $u_{i+1}$ ab.
    \end{itemize}
\end{frame}

\begin{frame}{Einschrittverfahren: Lokaler Diskretisierungsfehler}
    \begin{itemize}
        \item<1-> Der \textbf{lokale Diskretisierungsfehler} für das Einschrittverfahren \eqref{einschrittverfahren} ist
        definiert als
        \begin{align}
            \label{lokaler-fehler}
            \tau(\hat{t},h) := \frac{x(\hat{t} + h)-u_1(\hat{t})}{h}
        \end{align}
        für $\hat{t} \in [t_0, t_+a]$ und $u_1(\hat{t})=x(\hat{t})+h\Phi(\hat{t},x(\hat{t}),h,f)$.
        \item<2-> Das Einschrittverfahren heißt \textbf{konsistent} (von der Ordnung p), wenn für alle
        $\hat{t}\in(t_0,t_0+a]$
        \[
            \lim_{h \rightarrow 0 } \tau(\hat{t}, h)=0 \quad (\tau(\hat{t},h) = \mathcal{O}(h^p) \text{ für } h \rightarrow 0)
        \]
        gilt.
    \end{itemize}
\end{frame}

\begin{frame}{Einschrittverfahren: Globaler Diskretisierungsfehler}
    \begin{itemize}
        \item<1-> Der \textbf{globale Diskretisierungsfehler} für das Einschrittverfahren \eqref{einschrittverfahren} ist
        definiert als
        \begin{align}
            \label{globaler-fehler}
            e(\hat{t}, h) := x(\hat{t}) - u_i.
        \end{align}

        \item<2-> Das Einschrittverfahren heißt \textbf{konvergent} (von der Ordnung p), wenn für alle
        $\hat{t}\in(t_0,t_0+a]$
        \[
            \lim_{h \rightarrow 0 } e(\hat{t}, h)=0 \quad (e(\hat{t},h) = \mathcal{O}(h^p) \text{ für } h \rightarrow 0)
        \]
        gilt.
    \end{itemize}
\end{frame}

\begin{frame}{Einschrittverfahren: Fehlerabschätzung}
    \begin{itemize}
        \item<1-> Für ein Einschrittverfahren \eqref{einschrittverfahren} mit Lipschitz-stetiger Inkrementfunktion im
        zweiten Argument gilt die Abschätzung
        \[
            \left\lVert e(t_i,h) \right\rVert_2 \leq \left( \left\lVert e_0 \right\rVert_2 + (t_i-t_0)\tau_h \right)
            e^{L(t_i-t_0)}, \quad i = 1, \dots, N
        \]
        mit $e_0 = x(t_0)-u_0$ und $\tau_h= \max\limits_{k=1,\dots,N} \left\lVert \tau(t_k,h) \right\rVert_2 $.
        \item<2-> Die Abschätzung besagt, dass die Konsistenz des Einschrittverfahrens bereits die Konvergenz impliziert.
    \end{itemize}
\end{frame}

\subsection{Runge-Kutta-Verfahren}

\begin{frame}{Explizite Runge-Kutta-Verfahren}
    \begin{itemize}
        \item<1-> Ein m-stufiges, explizites \textbf{Runge-Kutta-Verfahren} ist ein Einschrittverfahren der Form
        \begin{align*}
            u_{i+1}&=u_i+h\sum_{l=1}^{m}b_{l}k_{l,i}\\
        \end{align*}
        mit den Stufen
        \begin{align*}
            k_{1,i} &= f(t_i,u_i)\\
            k_{l,i} &= f(t_i+c_lh, u_i + h\sum_{j=1}^{l-1}a_{lj}k_{j,i} ), \quad l=2,\dots,m.
        \end{align*}
    \end{itemize}
\end{frame}

\begin{frame}{Implizite Runge-Kutta-Verfahren}
    \begin{itemize}
        \item<1-> Ein m-stufiges, implizites Runge-Kutta-Verfahren hat die Form
        \begin{align*}
            k_{1,i} &= f(t_i+c_1h, u_i + h(a_{11}k_{1,i} + \dots + a_{1m} k_{m,i})) \\
            &\vdots\\
            k_{m,i} &= f(t_i+c_mh, u_i + h(a_{m1}k_{1,i} + \dots + a_{mm} k_{m,i}))\\
            u_{i+1} &= u_i + h(b_1 k_{1,i} + \dots + b_m k_{m,i})
        \end{align*}
    \end{itemize}
\end{frame}

\begin{frame}{Butcher-Tabellen}
    \begin{itemize}
        \item<1-> Die Koeffizienten $c,b,A$ der Runge-Kutta-Verfahren werden in Butcher-Tabellen zusammengefasst
        \begin{center}
            \begin{tabular}{c | c}
                c & A               \\
                \hline
                & $b^{\intercal}$
            \end{tabular}
        \end{center}
        \item Das \textbf{Verfahren von Heun} der Form
        \[
            u_{i+1}=h\left(\frac{1}{2}f(t_i,u_i)+\frac{1}{2}f(t_i+h,u_i+hf(t_i,u_i))\right)
        \]
        hat die Butcher-Tabelle
        \begin{center}
            \begin{tabular}{c | c c}
                0   & 0             & 0             \\
                $1$ & 1             & 0             \\
                \hline
                & $\frac{1}{2}$ & $\frac{1}{2}$
            \end{tabular}
        \end{center}
    \end{itemize}
\end{frame}

%\begin{frame}{Runge-Kutta-Verfahren: Konsistenz}
%    \begin{itemize}
%        \item<1-> Ein explizites Runge-Kutta-Verfahren ist invariant gegenüber Autonomisierung, wenn gilt:
%        es approximiert das Anfangswertproblem \eqref{awp-erste-ordnung} mit der Näherungslösung $u_i$
%        $\Leftrightarrow$ es approximiert das autonomisierte Anfangswertproblem
%        \begin{alignat*}{3}
%            z^{\prime} &= 1, \qquad &z(t_0) &= t_0, \\
%            x^{\prime} &= f(z,x),\qquad  &x(t_0) &= x_0 \\
%        \end{alignat*}
%        mit der Näherungslösung $v_i := \left[ \begin{matrix} t_i\\ u_i\\ \end{matrix} \right]$.
%        \item<2-> Ein explizites Runge-Kutta-Verfahren mit den Bedingungen
%        \[
%            \sum_{j=1}^{m} b_j = 1 \qquad \text{und} \qquad c_l=\sum_{j=1}^{l-1} a_{lj}, \quad l = 1, \dots, m
%        \]
%        ist invariant gegenüber Autonomisierung.
%    \end{itemize}
%\end{frame}

\begin{frame}{Runge-Kutta-Verfahren: Konsistenz}
    \begin{itemize}
        \item<1-> Für die maximale erreichbare Konsistenzordnung eines expliziten Runge-Kutta-Verfahrens mit m Stufen
        gilt
        \[
            p \leq m.
        \]
        \item<2-> Die Konsistenzordnung und die Stufenanzahl haben die Zusammenhänge
        \begin{center}
            \begin{tabular}{ c | c | c }
                Stufe $m$ & 1 \quad 2 \quad 3 \quad 4 \quad 5 \quad 6 \quad 7 \quad 8 \quad 9 & \quad $\geq 9$  \\
                \hline
                Ordnung p & 1 \quad 2 \quad 3 \quad 4 \quad 4 \quad 5 \quad 6 \quad 6 \quad 7 & \quad  $\leq m-2$
            \end{tabular}
        \end{center}
    \end{itemize}
\end{frame}

\begin{frame}{Runge-Kutta-Verfahren: Schrittweitensteuerung}
    \begin{itemize}
        \item<1-> Zur Steuerung der Schrittweite \textbf{h} werden \textbf{eingebettete} Runge-Kutta-Verfahren der Ordnung
        $p$ und $p+1$ verwendet, um den lokalen Fehler abschätzen zu können.
        \item<2-> Bei zu großem Fehler wird eine kleinere Schrittweite $h_{\text{neu}}<h_{\text{alt}}$ gewählt und der
        aktuelle Schritt wird wiederholt.
        \item<3-> Bei zu kleinem Fehler wird die Schrittweite mit Hilfe einer Hochschaltungsbeschränkung $q$ angepasst.
        Hierbei setzt man $h_{\text{neu}}=\min(qh_{\text{alt}}, h_{\text{max}})$ mit dem Hochschaltungsfaktor $q>1$.
        \item<4-> Der aktuelle Schritt wird akzeptiert, wenn der geschätzte Fehler kleiner als die Toleranz $\epsilon>0$
        ist.
    \end{itemize}
\end{frame}

\subsection{Mehrschrittverfahren}

\begin{frame}{Mehrschrittverfahren}
    \begin{itemize}
        \item<1-> Neben Einschrittverfahren gibt es auch \textbf{Mehrschrittverfahren} bzw. \textbf{k-Schrittverfahren}
        der Form
        \begin{align*}
            &\text{Startwerte } u_0, \dots, u_{k-1} \in \mathbb{R}^n, \nonumber \\
            & \sum_{j=0}^{k} \alpha_j u_{i+j} = h \phi(t_i, u_i, \dots, u_{i+k},h,f), \quad i=0,\dots,N-k
        \end{align*}
        mit $\alpha_k\neq0$.
        \item<2-> Falls die Inkrementfunktion $\Phi$ unabhängig von $u_{i+k}$ ist, nennt man das k-Schrittverfahren
        explizit, sonst implizit.
    \end{itemize}
\end{frame}

\begin{frame}{Lineare Mehrschrittverfahren}
    \begin{itemize}
        \item<1-> Ein lineares k-Schrittverfahren hat die Form
        \begin{align}
            \label{k-schritt-linear}
            \sum_{j=0}^{k} \alpha_j u_{i+j} = h \sum_{j=0}^{k} \beta_j f(t_{i+j}, u_{i+j}), \quad i = 0, \dots, N-k.
        \end{align}
        \item<2-> Für $\beta_k=0$ ist das Verfahren explizit, sonst implizit.
        \item<3-> Ein lineares k-Schrittverfahren vom Typ (r,l) hat die Form
        \begin{align*}
            u_{i+k}-u_{i+r-l} = h \sum_{j=0}^{r} \beta_j^{(r,l)} f(t_{i+j},u_{i+j}),
        \end{align*}
        mit
        \[
            \beta_j^{(r,l)} := \int\limits_{-l}^{k-r} \prod\limits_{\underset{q \neq j}{q=0}}^{r} \frac{(r-q+x)}{(j-q)}\ dx.
        \]
    \end{itemize}
\end{frame}

\begin{frame}{Mehrschrittverfahren: Konsistenz}
    \begin{itemize}
        \item<1-> Der lokale Diskretisierungsfehler eines Mehrschrittverfahrens ist gegeben durch
        \[
            \tau(t_i,h) = \frac{1}{h} \sum_{j=0}^{k} \alpha_j x(t_i + jh) -
            \phi (t_i,x(t_i), x(t_i+h), \dots, x(t_i+kh),h,f).
        \]
        \item<2-> Das Mehrschrittverfahren ist konsistent von der Ordnung $p\geq 1$, wenn für alle f mit
        stetiger und beschränkter Ableitung im zweiten Argument bis zur Ordnung $p$ gilt, dass
        $\tau(t_i,h) = \mathcal{O}(h^p)$.
        \item<3-> Das lineare Mehrschrittverfahren vom Typ (r,l) hat die Konsistenzordnung $p=r+1$, falls die rechte
        Seite $f$ (r+1)-mal stetig-differenzierbar auf $D \times K$, $K\subseteq \mathbb{R}^n$, ist.
    \end{itemize}
\end{frame}

\begin{frame}{Mehrschrittverfahren: Konsistenz}
    \begin{itemize}
        \item<1-> Für ein lineares \textit{k-Schrittverfahren} sind die Polynome
        \[
            \rho(z) := \sum_{j=0}^{k} \alpha_j z^j \qquad \text{und} \qquad \sigma(z) := \sum_{j=0}^{k} \beta_j z^j
        \]
        definiert.
        \item<2-> Das Polynom $\rho$ nennt man das erste charakteristische Polynom und $\sigma$ das zweite charakteristische
        Polynom des gegebenen Verfahrens.
    \end{itemize}
\end{frame}

\begin{frame}{Mehrschrittverfahren: Konsistenz}
    Für ein lineare k-Schrittverfahren sind folgende Aussagen äquivalent:
    \begin{align*}
        &\text{1. Das lineare k-Schrittverfahren ist konsistent von der Ordnung p.}\\
        &\text{2. Für das Anfangswertproblem }
        x^{\prime} = x, \text{ }
        x(0) = 1
        \text{ gilt } \tau(t_i,h) = \mathcal{O}(h^p).\\
        &\text{3. } \frac{\rho(z)}{\ln(z)} - \sigma(z) \text{ hat eine $p$-fache Nullstelle in } z= 1,\\
        &\text{ d.h. }
        \frac{\rho(z)}{\ln(z)} - \sigma(z) = \mathcal{O}((z-1)^p) \quad \text{ für } z \rightarrow 0.\\
        &\text{4. } \sum_{j=0}^{k} \alpha_j = 0 \quad \text{ und} \quad
        \sum_{j=0}^{k} j^m \alpha_j = \sum_{j=0}^{k} mj^{m-1} \beta_j,  \quad \text{für } m=0,1 \dots, p.
    \end{align*}
\end{frame}

\begin{frame}
    \begin{itemize}
        \item<1-> Ein Mehrschrittverfahren ist konvertgent (von der Ordnung p), also
        \[
            \left\lVert x(t_i) - u_i \right\rVert_2 = \mathcal{O}(h^p), \text{ für } t_i=t_0+ih \in [t_0,t_0+a],
        \]
        wenn es stabil TODO:cite stabilität und konsistent ist.
    \end{itemize}
\end{frame}

\begin{frame}{Mehrschrittverfahren für steife Differentialgleichungen}
    \begin{itemize}
        \item<1-> Eine lineare Differentialgleichung $x^{\prime}(t)=Ax(t)+g(t)$ heißt steif, wenn gilt:
        \begin{align*}
            1.& \quad Re(\lambda_j(A)) < 0, \qquad \text{ für alle Eigenwerte } \lambda_j(A) \text{ von A, }\\
            2.& \quad \min\limits_{1 \leq j \leq n} Re(\lambda_j(A)) \ll \max\limits_{1 \leq j \leq n} Re(\lambda_j(A)).
        \end{align*}
        \item<2-> Für steife Differentialgleichungen eignen sich \textbf{Rückwärtsdifferenzenverfahren}
        (engl: backward differentiation formula oder BDF) besonders gut zur Approximation
        \item<3-> BDF Verfahren sind implizite Mehrschrittverfahren und haben die Form
        \[
            \sum_{j=0}^{k}\alpha_j u_{i+j} = h f(t_{i+k}, u_{i+k})
        \]
        mit
        \[
            \alpha_j = h \mathcal{L}^{\prime}_{i+j} (t_{i+k}), \quad j=0,\dots,k.
        \]
    \end{itemize}
\end{frame}

\begin{frame}{Beispiel: Steife Differentialgleichung}
    \begin{itemize}
        \item<1-> Gegebenes Anfangswertproblem
        \begin{align}
            \label{stiff}
            x_{1}^{\prime} &= \frac{1}{2} ((\lambda_1 + \lambda_2)x_1 + (\lambda_1 - \lambda_2)x_2), \nonumber \\
            x_{2}^{\prime} &= \frac{1}{2} ((\lambda_1 - \lambda_2)x_1 + (\lambda_1 + \lambda_2)x_2), \\
            x_1(0) &= c_1 + c_2, \quad x_2(0) = c_1 - c_2 \nonumber
        \end{align}
        mit $\lambda_1, \lambda_2 < 0$ und $c_1,c_2 \in \mathbb{R}$.
        \item<2-> Die Matrix
        \[
            A = \frac{1}{2}
            \begin{bmatrix}
                \lambda_1 + \lambda_2 & \lambda_1 - \lambda_2 \\
                \lambda_1 - \lambda_2 & \lambda_1 + \lambda_2
            \end{bmatrix}
        \]
        hat die Eigenwerte $\lambda_1$ und $\lambda_2$.
    \end{itemize}
\end{frame}

\begin{frame}{Beispiel: Steife Differentialgleichung}
    \begin{itemize}
        \item<1-> Für $\lambda_1=-100$ und $\lambda_2=-1$ ist \eqref{stiff} steif.
        \item<2-> Die Lösung für \eqref{stiff} ist explizit gegeben mit
        \[
            x(t) =
            \begin{bmatrix}
                c_1 e^{\lambda_1 t} + c_2 e^{\lambda_2 t} \\
                c_1 e^{\lambda_1 t} - c_2 e^{\lambda_2 t}
            \end{bmatrix}.
        \]
        \item<3-> Für $t \rightarrow \infty$ konvergiert der erste Summand schneller gegen $0$.
        \item<4-> Zur Approximation von \eqref{stiff} wird der explizite Euler verwendet
        \begin{align*}
            u_0 &= x_0 \\
            u_i &= u_{i-1} + hAu_{i-1}= \dots = (I + hA)^{i}u_0, \qquad i=1,\dots,N
        \end{align*}
    \end{itemize}
\end{frame}

\begin{frame}{Beispiel: Steife Differentialgleichung}
    \begin{itemize}
        \item<1-> Die numerische Lösung hat die Form
        \begin{align*}
            u_{i}=
            \begin{bmatrix}
                c_1 (1+h\lambda_1)^{i} + c_2 (1+h\lambda_2)^{i}\\
                c_1 (1+h\lambda_1)^{i} + c_2 (1+h\lambda_2)^{i}
            \end{bmatrix}.
        \end{align*}
        \item<2-> Die numerische Lösung und die exakte Lösung besitzen das gleiche Grenzverhalten, wenn
        $|1+h\lambda_1|<1$ und $|1+h\lambda_2|<1$ bzw. $h<\frac{2}{|\lambda_1|}$ und $h<\frac{2}{|\lambda_2|}$ gilt.
        \item<3-> $\lambda_1$ bestimmt über die Beschränkung der Schrittweite, da $|\lambda_2| \ll |\lambda_1|$ gilt.
        \begin{itemize}
            \item<1-> Jedoch hat der Summand mit $\lambda_1$ für größere Zeiten $t$ fast keinen Einfluss auf die Lösung.
        \end{itemize}
    \end{itemize}
\end{frame}