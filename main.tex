\documentclass[handout, 10pt]{beamer}
\usepackage[ngerman]{babel}
\usetheme{Malmoe}
\usecolortheme{seagull}

\usepackage{algorithm}
\usepackage{algpseudocode}
\usepackage{subcaption}
\usepackage{tikz}
\usepackage{graphicx}
\usepackage{pgfplots}
\usepackage{subfiles}
\usepackage{neuralnetwork}

\setbeamertemplate{navigation symbols}{}

\usepackage[backend=bibtex,style=numeric]{biblatex}
\usepackage{amsmath}
\usepackage{amsfonts} % verbose-ibid
\addbibresource{bibliography.bib}
\renewcommand*{\bibfont}{\tiny}

%-----------------------------------------------------------------------------------
%%% DOCUMENT-HEAD %%%
%-----------------------------------------------------------------------------------

% This info will be visible in the document head
\author{Alexandro Jedaidi}
\title{\textbf{Vergleich der Integrationsmethoden und der Methoden des maschinellen
Lernens für gewöhnliche Differentialgleichungen}}
\date{\today}

\begin{document}
    \begin{frame}
        \titlepage
    \end{frame}

    \begin{frame}
        \frametitle{Inhaltsverzeichnis}
        \tableofcontents
    \end{frame}

    \subfile{sections/problemstellung.tex}

    \subfile{sections/numerik.tex}


    \subfile{sections/MachineLearning.tex}


    \subfile{sections/anwendungsbeispiele.tex}

    \section{Fazit}

    \begin{frame}{Fazit}
        \begin{itemize}
            \item<1-> Numerische Verfahren liefern im Allgemeinen viel bessere Ergebnisse.
            \item<2-> Die numerischen Verfahren haben außerdem geringere Rechenzeiten.
            \item<3-> Bei bereits optimierten neuronalen Netzen gibt es jedoch Vorteile.
            \begin{itemize}
                \item<1-> Das neuronale Netz simuliert eine Funktion, also sind Funktionsauswertungen möglich.
                \item<2-> Die Auswertungen können unter Verwendung leistungsschwacher Geräte stattfinden, sofern ein
                trainiertes neuronales Netz vorliegt.
            \end{itemize}
        \end{itemize}
    \end{frame}

\end{document}